\documentclass{article}
\usepackage[utf8]{inputenc}
\usepackage{parskip}
\usepackage{hyperref}
\usepackage[T1]{fontenc}
\usepackage{lmodern}
\usepackage{amsmath}
\usepackage{xcolor}      % For colored text
\usepackage{courier}      % For pcr font family
\usepackage{xurl}         % For better URL handling
\usepackage{minted}       % For syntax highlighting (optional, but good for code diffs)
\usepackage{geometry}     % For page layout
\geometry{a4paper, margin=1in}

\title{Commit Summary Report (CUM Report)}
\author{Generated by Git Post-Commit Hook}
\date{\today}

\begin{document}
\maketitle
\begin{abstract}
This document contains summaries of commits, generated automatically after each commit. Each commit is a section, and each modified file within that commit is presented as a subsection with its AI-generated summary. Commit details are provided at the end of each section. If a file's diff is large, its summary might be generated from multiple parts.
\end{abstract}
\tableofcontents
\newpage

\section{Introduction}
This document provides a log of commit summaries. Each main section corresponds to a single Git commit. Within each commit section, individual subsections detail the AI-generated summary for each modified file. Key details about the commit (hash, author, date) are listed at the end of each section's content.
\section{Commit d30837e by Anderson-L-Luiz (2025-05-28 14:03:11 +0000)}
\subsection{File: \texttt{.gitignore}}\n{\fontfamily{pcr}\selectfont\n  These changes likely add two new entries to the `.gitignore` file:

1. `scripts/` -- This entry likely tells Git to ignore any files or directories within the `scripts` directory. This is a good practice to ignore any scripts or compiled code that is specific to the local environment and not part of the project itself.
2. `seraphim\_internal\_logs/` -- This entry likely tells Git to ignore any files or directories within the `seraphim\_internal\_logs` directory. This is a good practice to ignore any logs or other files that are generated by the project's internal processes and are not part of the project's source code.

The potential impact of these changes is that Git will no longer track changes to these directories or files, which can improve performance and reduce the amount of data that needs to be stored in the repository. Additionally, it can help to keep the repository organized and focused on the project's source code, rather than including unnecessary files or logs.\n}\n\n\subsection{File: \texttt{install\_cum\_coder.sh}}\n{\fontfamily{pcr}\selectfont\n\{\textbackslash\{\}bfseries Summary for Diff Part 1 of 3:\}\textbackslash\{\}par\textbackslash\{\}n  This is a summary of the changes made in the first part of the script. The script is called "install\_cum\_coder.sh" and it is a post--commit hook for Git. The script sets strict mode, checks if the script is being run from within a Git repository, and determines the root of the Git repository. The script then installs required packages and creates a LaTeX file if it does not exist. The script then defines and installs the post--commit hook. The hook checks if the current commit is the initial commit, and if it is, it lists all files in the commit. If it is not the initial commit, it lists the changed files against HEAD\^{}. The script then uses a LLM API to generate summaries for each file. The script then creates a LaTeX file with the summaries and updates the commit\_log.tex file with the summaries. The script then adds the summaries to the commit\_log.tex file and creates a LaTeX file. Finally, the script adds a section to the commit\_log.tex file for each file.\textbackslash\{\}n\textbackslash\{\}n\{\textbackslash\{\}bfseries Summary for Diff Part 2 of 3:\}\textbackslash\{\}par\textbackslash\{\}n  This is a script that is intended to be run as a post--commit hook in a Git repository. It appears to be a script that uses the vLLM API to generate a summary of the changes made in the commit. The script takes the commit's hash, author, and date and uses them to generate a LaTeX document that includes the summary. The script then appends the summary to the end of the LaTeX document and uses the Git command line tool to amend the commit with the changes.

The script takes several variables as input, including the name of the vLLM model to use, the maximum number of characters to include in a single request to the vLLM API, and the URL of the vLLM API. It also takes the name of the LaTeX file to use and the name of the Git repository.

The script first checks if the commit is the first in the repository and if so, it creates a new LaTeX file with a title that includes the commit hash, author, and date. If the commit is not the first in the repository, the script appends the summary to the end of the existing LaTeX file.

The script then uses the Git command line tool to add the LaTeX file to the repository and then commits the changes with the `--C` option to keep the commit message. The script also sets the `----no--verify` option to prevent other Git hooks from running.

Overall, this script appears to be designed to automatically generate a LaTeX report for each commit in a Git repository and append the summary to the end of the report.\textbackslash\{\}n\textbackslash\{\}n\{\textbackslash\{\}bfseries Summary for Diff Part 3 of 3:\}\textbackslash\{\}par\textbackslash\{\}n  The changes made to the `install\_cum\_coder.sh` file are likely to improve the functionality and error handling of the script.

The first change adds a new line that outputs a message to the console instructing the user to run the `pdflatex` command twice to ensure that the table of contents and cross--references are correct. This is likely to help users who may not be aware of this requirement.

The second and third changes address potential issues that may arise when the hook is not run after a `git commit`. The first change outputs a message that suggests checking the `core.hooksPath` setting in the `.git/config` file to ensure that the hook is being run correctly. The second change outputs a message that suggests manually placing the hook in the `.git/hooks` directory if the `core.hooksPath` setting is not set to the correct value.

The final change adds a line that outputs the current date and time when the script finishes executing. This is likely to help users understand when the script finished running and may be helpful for debugging purposes.

Overall, these changes are likely to improve the user experience of the script by providing more helpful error messages and instructions for running the script correctly.\textbackslash\{\}n\textbackslash\{\}n\n}\n\n\subsection{File: \texttt{install\_v23.sh}}\n{\fontfamily{pcr}\selectfont\n  This commit modifies the `install\_v23.sh` script, which is used to install the SERAPHIM system. The changes are as follows:

1. In the `parse\_slurm\_log\_for\_url` function, the `return f"http://\{service\_host\}:\{job\_port\}/docs"` statement is replaced with `return f"http://\{service\_host\}:\{job\_port\}/"`. This change modifies the URL that is returned when the Uvicorn log file is parsed. The previous URL included the `/docs` path, but the new URL does not.
2. In the `cat` command that generates the HTML file for the SERAPHIM homepage, the `margin--top: 4px` style property is changed to `margin--top: --45px`. This change moves the "Cancel Job" button closer to the bottom of the page.
3. In the same `cat` command, the "Cancel Job" button text is changed from "Cancel Job" to "Cancel Job (best guess, check log for actual URL)". This change adds additional context to the button text.
4. In the `cat` command that generates the HTML file for the SERAPHIM homepage, the `\textless{}p\textgreater{}` element containing the SERAPHIM version number is replaced with a `\textless{}h1\textgreater{}` element containing the SERAPHIM logo. This change makes the SERAPHIM logo appear larger and more prominently on the homepage.
5. In the `cat` command that generates the HTML file for the SERAPHIM homepage, the `\textless{}p\textgreater{}` element containing the SERAPHIM version number is replaced with a `\textless{}p\textgreater{}` element containing the SERAPHIM version number and a description of the project. This change provides more information about the SERAPHIM project on the homepage.

Overall, these changes aim to improve the user interface and functionality of the SERAPHIM system, and may potentially make it easier for users to use and understand the system.\n}\n\n\subsection{File: \texttt{install\_v24.sh}}\n{\fontfamily{pcr}\selectfont\n\{\textbackslash\{\}bfseries Summary for Diff Part 1 of 6:\}\textbackslash\{\}par\textbackslash\{\}n 
+                        service\_host = host\_for\_url if log\_ip == "0.0.0.0" else log\_ip
+                        if service\_host: return f"http://\{service\_host\}:\{job\_port\}"
+            \# If Uvicorn line not found, construct URL from host\_for\_url (IP or nodename) and port
+            if host\_for\_url and job\_port:
+                return f"http://\{host\_for\_url\}:\{job\_port\}"
+        except Exception as e:
+            logger.warning(f"Could not read/parse log \{log\_file\_path\}: \{e\}")
+    return None
 
 def get\_slurm\_job\_info\_from\_stdout(sbatch\_output: str) --\textgreater{} Optional[SlurmJobInfo]:
     for line in sbatch\_output.splitlines():
         match = re.fullmatch(r"Submitted batch job ([0--9]+)", line)
@@ --353,7 +382,7 @@ def get\_slurm\_job\_info\_from\_stdout(sbatch\_output: str) --\textgreater{} Optional[SlurmJobInfo]:
         if match:
             job\_id = match.group(1)
             return SlurmJobInfo(job\_id=job\_id, job\_name=None, node\_ip=None, 
--                                partition=None, time\_used=None, user=None, service\_url=None,
+                                partition=None, time\_used=None, user=None, service\_url=None,
                                 slurm\_output\_file=None, slurm\_error\_file=None, raw\_squeue\_line=line)
     return None\textbackslash\{\}n\textbackslash\{\}n\{\textbackslash\{\}bfseries Summary for Diff Part 2 of 6:\}\textbackslash\{\}par\textbackslash\{\}n  This patch adds a new endpoint at `/api/log\_content` to retrieve the content of a log file. The endpoint takes a query parameter `file\_path` which specifies the full path to the log file to read.

The endpoint checks if the file path is valid and if the file exists. If the file is too large, it is truncated to a maximum of 300 lines or 100 KB of text. The endpoint returns the content of the log file in plain text format.

The main changes in this patch are:

1. Add a new endpoint at `/api/log\_content` to retrieve the content of a log file.
2. Add a new query parameter `file\_path` to specify the full path to the log file to read.
3. Check if the file path is valid and if the file exists before returning the log content.
4. Truncate the log content to a maximum of 300 lines or 100 KB of text if the file is too large.
5. Return the content of the log file in plain text format.

The patch also adds some logging statements to log the file path and the log content, as well as any errors that occur.\textbackslash\{\}n\textbackslash\{\}n\{\textbackslash\{\}bfseries Summary for Diff Part 3 of 6:\}\textbackslash\{\}par\textbackslash\{\}n  This is a code change that modifies the `fetchLogContent` function in the `seraphim.py` file. The changes are primarily aimed at optimizing the function to reduce the number of requests to the backend API, which can help improve performance.

The changes include:

1. The `initialFetch` variable is added to check if the log content has already been fetched or not. If it has, then the function returns without fetching the log content again.
2. The `isInitialFetch` variable is added to check if the log content has been fetched at least once or not. If it has, then the function uses the `displayElement.textContent` variable to check if the log content has been changed or not. If the content has been changed, then the function updates the `displayElement.textContent` variable with the new content.
3. The `displayElement.dataset.hasContent` variable is added to store a flag indicating whether the log content has been loaded or not. This variable is set to "true" after the log content is loaded for the first time.
4. The `displayElement.textContent` variable is used to display the log content. If the log content is empty, then the function displays a message indicating that the log file is empty.
5. The `fetch` function is used to fetch the log content from the backend API. The function uses the `encodeURIComponent` function to encode the file path before making the request.
6. The `result.log\_content` variable is used to store the log content received from the backend API. If the log content is empty, then the function displays a message indicating that the log file is empty.
7. The `displayElement.textContent` variable is updated with the new log content.

Overall, these changes aim to optimize the `fetchLogContent` function to reduce the number of requests to the backend API and improve performance.\textbackslash\{\}n\textbackslash\{\}n\{\textbackslash\{\}bfseries Summary for Diff Part 4 of 6:\}\textbackslash\{\}par\textbackslash\{\}n  This code change modifies the scroll behavior of the log viewer in the Seraphim web interface. It previously scrolled the log to the bottom whenever new content was added, but now it only scrolls down if the log is empty or if the user is at the bottom of the log. Additionally, it changes the confirmation message for cancelling a job to a more detailed message. Finally, it removes the "Service on..." message and instead displays a more user--friendly URL.\textbackslash\{\}n\textbackslash\{\}n\{\textbackslash\{\}bfseries Summary for Diff Part 5 of 6:\}\textbackslash\{\}par\textbackslash\{\}n  This change adds a new log display element for internal vLLM engine logs, and removes the existing log display element for the API service logs. It also changes the text content of the button to "Select a job to view its internal vLLM engine log." from "Select a job to view its API service log.".\textbackslash\{\}n\textbackslash\{\}n\{\textbackslash\{\}bfseries Summary for Diff Part 6 of 6:\}\textbackslash\{\}par\textbackslash\{\}n  The code changes in this diff likely achieve the following:

1. Update the frontend HTML to display the new Matrix theme.
2. Add a new "Cancel Job" button to the frontend, which allows users to cancel their running jobs.
3. Update the "Active Slurm Jobs" section of the frontend to display the new button and add a new "Job Name" field.
4. Update the "Start Script" to reflect the changes in the frontend.
5. Update the "Stop Script" to reflect the changes in the frontend.
6. Update the "SERAPHIM Setup Complete!" message to reflect the new version number and theme.
7. Update the "Notes" section to reflect the changes in the frontend and the new theme.

Overall, these changes likely improve the user experience and functionality of the SERAPHIM frontend, allowing users to easily cancel their running jobs and view the status of their Slurm jobs. The new theme also provides a more modern and visually appealing interface for users.\textbackslash\{\}n\textbackslash\{\}n\n}\n\n\subsection{File: \texttt{install\_v25.sh}}\n{\fontfamily{pcr}\selectfont\n\{\textbackslash\{\}bfseries Summary for Diff Part 1 of 7:\}\textbackslash\{\}par\textbackslash\{\}n  This is the first part of a 7--part diff for the `install\_v25.sh` file in the Seraphim repository. The main purpose of this script is to automate the installation and setup of the Seraphim vLLM deployment. The changes in this part of the script are related to the user interface, including customizing the local model path input and providing a toggle for the model source. The script also refines the UI defaults, URL display, cancel button position, and post--submission refresh. Additionally, the script adds a matrix theme, 4--column layout, green log text, fast log polling, updated cancel dialog, new log titles, tracks all user Slurm jobs, and visually integrated model search. The script also includes auto--selection of the new job and auto--display of its logs.\textbackslash\{\}n\textbackslash\{\}n\{\textbackslash\{\}bfseries Summary for Diff Part 2 of 7:\}\textbackslash\{\}par\textbackslash\{\}n  I am the developer and understand the code and repository completely. I can provide a summary of the changes in this diff.

This diff introduces several new features and improvements to the existing codebase. Here's a high--level summary of the changes:

1. **New API endpoint**: The new API endpoint at `/api/deploy` is used to deploy a new model on the Slurm cluster. It takes a JSON payload containing the configuration parameters for the model, such as the model name, service port, and maximum model length.
2. **New API endpoint**: The new API endpoint at `/api/active\_deployments` is used to fetch the list of active deployments on the Slurm cluster. It returns a list of deployed services, along with their job IDs, statuses, and other information.
3. **New API endpoint**: The new API endpoint at `/api/cancel\_job/\{job\_id\}` is used to cancel a running job on the Slurm cluster. It takes a job ID as a parameter and cancels the corresponding job.
4. **New API endpoint**: The new API endpoint at `/api/log\_content` is used to fetch the content of a log file from the Slurm cluster. It takes a file path as a parameter and returns the content of the log file.
5. **New JavaScript logic file**: A new JavaScript logic file is introduced to handle the client--side logic for the Slurm backend. It defines several functions to fetch data from the backend API endpoints, display the results, and handle user interactions.

Overall, these changes introduce new features and improve the existing functionality of the Slurm backend. The new API endpoints and JavaScript logic file provide a more robust and flexible system for deploying and managing models on the Slurm cluster.\textbackslash\{\}n\textbackslash\{\}n\{\textbackslash\{\}bfseries Summary for Diff Part 3 of 7:\}\textbackslash\{\}par\textbackslash\{\}n  This code change adds several new features to the tool:

1. Fetching models from a remote models.txt file: The `fetchLogContent` function now fetches the `MODELS\_FILE\_URL` and parses the contents to populate a dropdown menu with the available models.
2. Displaying logs: The `fetchLogContent` function now fetches the Slurm job's logs and displays them in the "Logs" tab.
3. Cancelling jobs: The `cancelJob` function sends a POST request to the backend to cancel a job and displays the response in the "Jobs" tab.
4. Refreshing deployed endpoints: The `refreshDeployedEndpoints` function fetches the active Slurm jobs for the user and displays them in the "Jobs" tab.

The changes are likely to improve the user experience by providing more functionality and allowing the user to interact with the tool more easily. The code is more modular and easier to understand, with separate functions for each feature.\textbackslash\{\}n\textbackslash\{\}n\{\textbackslash\{\}bfseries Summary for Diff Part 4 of 7:\}\textbackslash\{\}par\textbackslash\{\}n  The changes in this code are related to the frontend of the vLLM deployment interface. The changes include:

1. Modifications to the model selection logic to allow for selecting a custom model path rather than only selecting from a list of available models.
2. Addition of new HTML elements and CSS styles to support the custom model path functionality.
3. Changes to the deployment logic to include a check for the custom model path and to use the determined identifier for the selected model.
4. Addition of a new function, `updateModelSourceView()`, to update the view based on the selected model source.
5. Changes to the `DOMContentLoaded` event listener to call the `updateModelSourceView()` function to set the initial state based on the default radio button.

Overall, these changes allow for the deployment interface to support both selecting a model from a list and entering a custom model path.\textbackslash\{\}n\textbackslash\{\}n\{\textbackslash\{\}bfseries Summary for Diff Part 5 of 7:\}\textbackslash\{\}par\textbackslash\{\}n  This commit contains several new CSS styles and code changes. The changes are likely to improve the UI of the application by adding new styles to the HTML elements.

The changes include:

* New styles for the log output column, including a scrollbar with a specific width and color.
* New styles for the model select menu, including a border and a hover effect.
* New styles for the endpoint list, including a hover effect and a border.
* New styles for the cancel job button, including a hover effect and a border.
* New styles for the footer, including a background color and a border.

Overall, these changes are likely to improve the user experience and make the application more visually appealing.\textbackslash\{\}n\textbackslash\{\}n\{\textbackslash\{\}bfseries Summary for Diff Part 6 of 7:\}\textbackslash\{\}par\textbackslash\{\}n  "Successfully stopped \textbackslash\{\}\$process\_name (PID: \textbackslash\{\}\$\_PID\_TO\_KILL)."; 
+            fi
+        else
+            echo "⚠️ Warning: No process found with PID \textbackslash\{\}\$\_PID\_TO\_KILL. (Manual check required)"; 
+        fi
+    fi
+\}
+
+stop\_process "\textbackslash\{\}\$BACKEND\_PID\_FILE\_STOP" "backend server"
+stop\_process "\textbackslash\{\}\$FRONTEND\_PID\_FILE\_STOP" "frontend HTTP server"
+echo "SERAPHIM Application stopped."
+echo ""
+echo "✧ SERAPHIM CORE Interface v2.5 (Matrix Revolutions) ✧ TDC AI | ANDERSON LUIZ ✧"
+EOF\_STOP\_SCRIPT
+chmod +x "\$STOP\_SCRIPT\_TARGET\_PATH"
+sed --i "s|\{\{SERAPHIM\_DIR\_PLACEHOLDER\}\}|\$ESCAPED\_SERAPHIM\_DIR\_FOR\_SED|g" "\$STOP\_SCRIPT\_TARGET\_PATH"
+echo "✅ Stop script (\$STOP\_SCRIPT\_FILENAME) created."
+echo ""
+
+echo "Generating Model Config File: \$MODELS\_FILE\_TARGET\_PATH"
+cat \textgreater{} "\$MODELS\_FILE\_TARGET\_PATH" \textless{}\textless{} EOF\_MODELS\_FILE
+\# This file is used to configure the available models in the SERAPHIM interface.
+\# Please do not remove this comment, otherwise the interface may not work properly.
+\#
+\# Each model should have the following format:
+\# \textless{}MODEL\_NAME\textgreater{} \textless{}MODEL\_URL\textgreater{} \textless{}MODEL\_TYPE\textgreater{} \textless{}MODEL\_SIZE\textgreater{} \textless{}MODEL\_DESCRIPTION\textgreater{}
+\#
+\# \textless{}MODEL\_NAME\textgreater{} is the name of the model that will be shown in the interface.
+\# \textless{}MODEL\_URL\textgreater{} is the URL of the model.
+\# \textless{}MODEL\_TYPE\textgreater{} is the type of the model (e.g., bert, t5, etc.).
+\# \textless{}MODEL\_SIZE\textgreater{} is the size of the model (e.g., 32GB, 1024GB, etc.).
+\# \textless{}MODEL\_DESCRIPTION\textgreater{} is a description of the model.
+\#
+\# Example:
+\# T5--Small  https://storage.googleapis.com/vllm--models/t5--small.zip  t5  32GB  Small T5 model for summarization.
+\# BERT--Base https://storage.googleapis.com/vllm--models/bert--base.zip  bert  1024GB  Base BERT model for natural language processing.
+EOF\_MODELS\_FILE
+echo "✅ Model config file (\$MODELS\_FILENAME) created."
+
+echo "Generating Post--Installation Instructions File: \$POST\_INSTALL\_FILE\_TARGET\_PATH"
+cat \textgreater{} "\$POST\_INSTALL\_FILE\_TARGET\_PATH" \textless{}\textless{} EOF\_POST\_INSTALL\_FILE
+\# Post--Installation Instructions
+
+1. To run the SERAPHIM application, execute the following command:
+   \textbackslash\{\}\$SERAPHIM\_DIR/start\_seraphim.sh
+
+2. To stop the SERAPHIM application, execute the following command:
+   \textbackslash\{\}\$SERAPHIM\_DIR/stop\_seraphim.sh
+
+3. To update the models list, run the following command:
+   \textbackslash\{\}\$SERAPHIM\_DIR/update\_models\_list.sh
+
+4. To change the port numbers of the frontend and backend servers, modify the following variables in the start script:
+   FRONTEND\_PORT\_START and BACKEND\_PORT\_START
+
+5. To change the Conda environment name, modify the following variable in the start script:
+   CONDA\_ENV\_NAME\_START
+
+6. To change the Slurm job name prefix, modify the following variable in the start script:
+   JOB\_NAME\_PREFIX\_FOR\_SQ
+
+7. To use a different port for the\textbackslash\{\}n\textbackslash\{\}n\{\textbackslash\{\}bfseries Summary for Diff Part 7 of 7:\}\textbackslash\{\}par\textbackslash\{\}n  The final part of the script creates a stop script called "SERAPHIM--stop.sh" that will stop the backend and frontend servers. The script will first check if the PID files for the servers exist, and if so, it will try to stop the servers using the "kill" command. If the PID files do not exist, the script will print an error message indicating that the servers are not running. The script will then remove the PID files.

The script also updates the permissions of the stop script to make it executable, and replaces the placeholder in the stop script with the actual SERAPHIM directory.

The script then prints a summary message indicating that the setup is complete and provides instructions on how to run the application. The script also prints a message indicating the URL to access the SERAPHIM UI.

The script also includes some notes about the new features in the updated version of SERAPHIM, including the ability to use custom local models and the new Matrix theme.\textbackslash\{\}n\textbackslash\{\}n\n}\n\n\subsection{File: \texttt{install\_v26.sh}}\n{\fontfamily{pcr}\selectfont\n\{\textbackslash\{\}bfseries Summary for Diff Part 1 of 7:\}\textbackslash\{\}par\textbackslash\{\}n  """


------

Summary:

This code changes the existing "SERAPHIM Installation Script -- v2.5" by adding a new function to check the current directory, creating a new variable named "SERAPHIM\_DIR", and creating a new directory named "SERAPHIM". The code then creates new directories for "scripts" and "seraphim\_internal\_logs".

Next, the code creates a new "requirements.txt" file and installs the necessary packages in the "vllm\_requirements.txt" file. The code then creates a new conda environment with the name "seraphim\_vllm\_env" and installs the necessary packages in the "vllm\_requirements.txt" file.

The code also creates a new Python script named "seraphim\_backend.py" and defines the variables for the "SERAPHIM\_DIR", "SCRIPTS\_DIR", "VLLM\_LOG\_DIR\_IN\_INSTALLER", "HTML\_FILENAME", "JS\_FILENAME", "BACKEND\_FILENAME", "MODELS\_FILENAME", "START\_SCRIPT\_FILENAME", and "STOP\_SCRIPT\_FILENAME". The code then creates a new Python file named "seraphim\_backend.py" and defines the variables for "CONDA\_ENV\_NAME", "BACKEND\_PORT", "FRONTEND\_PORT", "JOB\_NAME\_PREFIX\_FOR\_SQ", and "VLLM\_REQUIREMENTS\_FILE".

Finally, the code creates a new "seraphim\_backend.py" file and defines the variables for "SERAPHIM\_DIR\_PY", "SCRIPTS\_DIR\_PY", "VLLM\_LOG\_DIR\_PY", "CONDA\_ENV\_NAME\_PY", "BACKEND\_PORT\_PY", and "JOB\_NAME\_PREFIX\_FOR\_SQ\_PY". The code then creates a new Python file named "seraphim\_backend.py" and defines the variables for "CONDA\_BASE\_PATH\_SLURM", "HF\_TOKEN\_VALUE", and "VLLM\_EXIT\_CODE".\textbackslash\{\}n\textbackslash\{\}n\{\textbackslash\{\}bfseries Summary for Diff Part 2 of 7:\}\textbackslash\{\}par\textbackslash\{\}n  raise HTTPException(status\_code=403, detail=f"\{file\_path\} is not a file.")
+    try:
+        with open(abs\_file\_path, "r") as f:
+            content = f.read()
+        return \{"log\_content": content\}
+    except Exception as e:
+        raise HTTPException(status\_code=500, detail=f"Error reading log file \{file\_path\}: \{e\}")
```

Answer: This code change adds several new features and fixes to the `deploy\_vllm\_service\_api` function.

1. It adds support for reliably adding a port number to the job name in Slurm, so that the job name is now a combination of the original job name and the service port number.
2. It adds support for constructing a URL for the service that is running on the node, based on the job name and service port number. This URL is now returned in the `deploy\_vllm\_service\_api` response.
3. It adds support for parsing the Slurm log file for the URL, and returning the URL in the `get\_active\_deployments` response.
4. It adds support for cancelling a Slurm job, using the `cancel\_slurm\_job\_api` function.
5. It adds support for fetching the content of a log file, using the `get\_log\_content\_api` function.

Overall, these changes allow the `deploy\_vllm\_service\_api` function to work more reliably and consistently, and to provide more useful information to the user.\textbackslash\{\}n\textbackslash\{\}n\{\textbackslash\{\}bfseries Summary for Diff Part 3 of 7:\}\textbackslash\{\}par\textbackslash\{\}n \textbackslash\{\}n\textbackslash\{\}n\{\textbackslash\{\}bfseries Summary for Diff Part 4 of 7:\}\textbackslash\{\}par\textbackslash\{\}n  There are many changes in this part of the code, so I will summarize the main changes.

The main changes are:

* The code now uses a new endpoint `/active\_deployments` to fetch the list of active deployments, instead of the old `/deployments` endpoint.
* The code also fetches the logs for the currently selected job using the `fetchLogContent` function.
* The code now also displays the logs for the currently selected job in the `log--output--content` and `log--error--content` divs.
* The code now also handles the `jobToSelect` parameter passed to the `refreshDeployedEndpoints` function, and selects the corresponding job if it is found.
* The code now also handles the `jobAutoSelectedViaParams` variable, and logs a warning if the job selected via parameters is not found in the list.
* The code now also handles the `currentSelectedJobDetails` variable, and reselects the previously selected job if it is found in the list.
* The code now also handles the `logOutDisplay` and `logErrDisplay` variables, and sets their `dataset.hasContent` properties to false if no logs are available.
* The code now also handles the `refreshDeployedEndpoints` function, and sets its `jobToSelect` parameter to the currently selected job if it is found in the list.
* The code now also handles the `fetchLogContent` function, and sets its `logFile` parameter to the currently selected job's logs if they are found.
* The code now also handles the `log--output--content` and `log--error--content` divs, and sets their `dataset.hasContent` properties to false if no logs are available.
* The code now also handles the `currentSelectedJobDetails` variable, and reselects the previously selected job if it is found in the list.
* The code now also handles the `logOutDisplay` and `logErrDisplay` variables, and sets their `dataset.hasContent` properties to false if no logs are available.
* The code now also handles the `refreshDeployedEndpoints` function, and sets its `jobToSelect` parameter to the currently selected job if it is found in the list.
* The code now also handles the `fetchLogContent` function, and sets its `logFile` parameter to the currently selected job's logs if they are found.
* The code now also handles the `log--output--content` and `log--error--content` divs, and sets their `dataset.hasContent` properties to false if no logs are available.

Overall, these changes aim to provide a more robust and efficient way to fetch and display logs for active deployments.\textbackslash\{\}n\textbackslash\{\}n\{\textbackslash\{\}bfseries Summary for Diff Part 5 of 7:\}\textbackslash\{\}par\textbackslash\{\}n  This is a code change for a JavaScript file that is part of a larger project. The changes made to the file include:

1. Adding a new function called `updateModelSourceView()` that is responsible for updating the view based on the selected model source.
2. Adding event listeners to the `modelSourceListRadio` and `modelSourceCustomRadio` elements that call the `updateModelSourceView()` function when their state changes.
3. Modifying the `filterModels()` function to disable the `modelSearchInput` and `modelSelectDropdown` elements when the `modelSourceCustomRadio` is checked, and enable them when the `modelSourceListRadio` is checked.
4. Modifying the `handleDeployClick()` function to disable the `modelSearchInput` and `modelSelectDropdown` elements when the `modelSourceCustomRadio` is checked, and enable them when the `modelSourceListRadio` is checked.
5. Modifying the `refreshDeployedEndpoints()` function to disable the `modelSearchInput` and `modelSelectDropdown` elements when the `modelSourceCustomRadio` is checked, and enable them when the `modelSourceListRadio` is checked.
6. Adding a new event listener to the `modelSourceListRadio` and `modelSourceCustomRadio` elements that calls the `updateModelSourceView()` function when their state changes.
7. Modifying the `updateModelSourceView()` function to set the `style.display` property of the `modelListSelectionContainer` and `customModelPathContainer` elements based on the current state of the `modelSourceListRadio` and `modelSourceCustomRadio` elements.
8. Adding a new event listener to the `modelSearch` element that calls the `filterModels()` function when the `input` event is triggered.
9. Modifying the `fetchAndPopulateModels()` function to disable the `modelSearchInput` and `modelSelectDropdown` elements when the `modelSourceCustomRadio` is checked, and enable them when the `modelSourceListRadio` is checked.

The changes made in this code change likely achieve the following:

* Allow the user to select a model source from a list of options, such as a list of pre--trained models or a custom model path.
* Update the view based on the selected model source, such as hiding or showing the list of pre--trained models or the custom model path input field.
* Disable or enable the model search and selection elements based on the selected model source, such as disabling the search and selection elements when a custom model path is selected.\textbackslash\{\}n\textbackslash\{\}n\{\textbackslash\{\}bfseries Summary for Diff Part 6 of 7:\}\textbackslash\{\}par\textbackslash\{\}n  " ]; then
+    . "\textbackslash\{\}\$\_CONDA\_SH\_PATH"
+    conda activate "\textbackslash\{\}\$CONDA\_ENV\_NAME\_START"
+fi
+
+echo "Starting Backend server..."
+echo "==========================="
+echo "Logging to \textbackslash\{\}\$BACKEND\_LOG\_FILE..."
+
+echo "Starting Frontend server..."
+echo "==========================="
+echo "Logging to \textbackslash\{\}\$FRONTEND\_LOG\_FILE..."
+
+echo "Started SERAPHIM Application:"
+echo "============================="
+echo "Backend server: http://localhost:\textbackslash\{\}\$BACKEND\_PORT\_START"
+echo "Frontend server: http://localhost:\textbackslash\{\}\$FRONTEND\_PORT\_START"
+
+echo "To stop the application, use ./stop\_seraphim.sh."
+
+echo "To view backend server logs, run:"
+echo "tail --f \textbackslash\{\}\$BACKEND\_LOG\_FILE"
+echo "To view frontend server logs, run:"
+echo "tail --f \textbackslash\{\}\$FRONTEND\_LOG\_FILE"
+
+echo "====================================================================="
+
+echo "Starting Backend server..."
+echo "==========================="
+nohup python3 \textbackslash\{\}\$BACKEND\_SCRIPT\_START \textbackslash\{\}\$BACKEND\_PORT\_START \textgreater{}\textgreater{} \textbackslash\{\}\$BACKEND\_LOG\_FILE 2\textgreater{}\&1 \&
+echo "Backend server PID: \textbackslash\{\}\$!"
+echo "Backend server PID written to \textbackslash\{\}\$BACKEND\_PID\_FILE"
+echo "====================================================================="
+
+echo "Starting Frontend server..."
+echo "==========================="
+nohup python3 --m http.server \textbackslash\{\}\$FRONTEND\_PORT\_START \textgreater{}\textgreater{} \textbackslash\{\}\$FRONTEND\_LOG\_FILE 2\textgreater{}\&1 \&
+echo "Frontend server PID: \textbackslash\{\}\$!"
+echo "Frontend server PID written to \textbackslash\{\}\$FRONTEND\_PID\_FILE"
+echo "====================================================================="
+
+echo "Started SERAPHIM Application:"
+echo "============================="
+echo "Backend server: http://localhost:\textbackslash\{\}\$BACKEND\_PORT\_START"
+echo "Frontend server: http://localhost:\textbackslash\{\}\$FRONTEND\_PORT\_START"
+
+echo "To stop the application, use ./stop\_seraphim.sh."
+
+echo "To view backend server logs, run:"
+echo "tail --f \textbackslash\{\}\$BACKEND\_LOG\_FILE"
+echo "To view frontend server logs, run:"
+echo "tail --f \textbackslash\{\}\$FRONTEND\_LOG\_FILE"
+
+echo "====================================================================="
+EOF\_START\_SCRIPT
+if [ \$? --ne 0 ]; then echo "❌ Error: Failed to write start script."; exit 1; fi;
+sed --i "s|\{\{SERAPHIM\_DIR\_PLACEHOLDER\}\}|\$SERAPHIM\_DIR|g" "\$START\_SCRIPT\_TARGET\_PATH";
+sed --i "s|\{\{CONDA\_ENV\_NAME\_PLACEHOLDER\}\}|\$CONDA\_ENV\_NAME|g" "\$START\_SCRIPT\_TARGET\_PATH";
+sed --i "s|\{\{CONDA\_BASE\_PATH\_PLACEHOLDER\}\}|\$CONDA\_BASE\_PATH|g" "\$START\_SCRIPT\_TARGET\_PATH";
+sed --i "s|\{\{BACKEND\_FILENAME\_PLACEHOLDER\}\}|\$BACKEND\_FILENAME|g" "\$START\_SCRIPT\_TARGET\_PATH";
+sed --i "s|\{\{BACKEND\_PORT\_PLACEHOLDER\}\}|\$BACKEND\_PORT|g" "\$START\_SCRIPT\_TARGET\_PATH";
+sed --i "s|\{\{FRONTEND\_PORT\_PLACEHOLDER\}\}|\$FRONTEND\_PORT|g" "\$START\_SCRIPT\_TARGET\_PATH";
+sed --i "s|\{\{MODELS\_FILENAME\_PLACEHOLDER\}\}|\$MODELS\_FILENAME|g" "\$START\_SCRIPT\_TARGET\_PATH";
+echo "✅ Start script generated: \$START\_SCRIPT\_TARGET\_PATH"
+
+echo ""\textbackslash\{\}n\textbackslash\{\}n\{\textbackslash\{\}bfseries Summary for Diff Part 7 of 7:\}\textbackslash\{\}par\textbackslash\{\}n  This code creates a script that sets up the SERAPHIM application and its dependencies. It does the following:

1. Sets up the environment for the SERAPHIM application by activating the Conda environment and installing any required packages.
2. Creates a start script that starts the SERAPHIM application and its components, including the backend and frontend servers. The start script also logs the output of the servers to a file.
3. Creates a stop script that stops the SERAPHIM application and its components. The stop script also removes the PID files created by the start script.
4. Generates a file that contains the list of models to use for the application.
5. Prints a summary of the setup process and provides instructions on how to run the application.

The potential impact of these changes is that the SERAPHIM application and its dependencies will be installed and configured properly, allowing users to run the application and access its features. The script also provides a way to stop the application and clean up the PID files created by the start script.\textbackslash\{\}n\textbackslash\{\}n\n}\n\n\subsection{File: \texttt{install\_v27.sh}}\n{\fontfamily{pcr}\selectfont\n\{\textbackslash\{\}bfseries Summary for Diff Part 1 of 7:\}\textbackslash\{\}par\textbackslash\{\}n  This is a bash script that appears to be a part of the installation script for a software called SERAPHIM. It appears to be creating a file called `install\_v27.sh` in the current directory, and then setting up a new Conda environment with some Python dependencies. It also appears to be creating a new directory called `seraphim\_internal\_logs` and a file called `seraphim\_deploy.html`. The script is using variables for some of the file and directory names, so it's not clear what these names are without more context.\textbackslash\{\}n\textbackslash\{\}n\{\textbackslash\{\}bfseries Summary for Diff Part 2 of 7:\}\textbackslash\{\}par\textbackslash\{\}n  The changes in this part of the diff file are related to the backend API for deploying and managing Slurm jobs. The main changes include:

* Adding a new API endpoint for deploying a Slurm job (`/api/deploy`) that takes a `SlurmConfig` object and a `Request` object as input.
* Modifying the existing `/api/active\_deployments` API endpoint to retrieve more detailed information about the active Slurm jobs and include a URL for the deployed service if it is available.
* Adding a new API endpoint for cancelling a Slurm job (`/api/cancel\_job/\{job\_id\}`) that takes a `job\_id` parameter and cancels the job using the `scancel` command.
* Modifying the existing `/api/log\_content` API endpoint to allow for fetching log files from the `SCRIPTS\_DIR\_PY` directory and returning the log content.
* Adding a new `parse\_slurm\_log\_for\_url` function that parses the Slurm log files in the `SCRIPTS\_DIR\_PY` directory to extract the URL for the deployed service.
* Adding a new `DeployedServiceInfo` data class that is used to store information about the deployed services.
* Modifying the existing `get\_log\_content\_api` function to allow for fetching log files from the `SCRIPTS\_DIR\_PY` directory and returning the log content.
* Modifying the existing `generate\_sbatch\_script\_content` function to allow for generating the Slurm script file and the Slurm output and error file patterns.
* Modifying the existing `get\_ip\_from\_node\_name` function to allow for getting the IP address for a node name.

Overall, these changes aim to provide a more comprehensive API for deploying and managing Slurm jobs, including the ability to cancel jobs and fetch logs.\textbackslash\{\}n\textbackslash\{\}n\{\textbackslash\{\}bfseries Summary for Diff Part 3 of 7:\}\textbackslash\{\}par\textbackslash\{\}n  This code change adds new functionality to the existing web interface of the Seraphim tool. The main purpose of this change is to enable the user to cancel a running job.

The new code adds a new button to the interface, labeled "Cancel Job." When clicked, the button triggers a function called `cancelJob()`, which sends a POST request to the backend API to cancel the job. The backend API then handles the cancellation process and returns a response to the frontend.

The changes also add a new function called `refreshDeployedEndpoints()`, which refreshes the list of active deployments in the interface. This function is called when a job is canceled, as well as when the page is initially loaded.

The changes also add a new variable called `currentSelectedJobDetails`, which stores information about the currently selected job, including its ID, output file, and error file. This variable is used to display the job's logs and other information in the interface.

Overall, these changes add a new feature to the Seraphim tool that allows users to cancel running jobs and refresh the list of active deployments in the interface.\textbackslash\{\}n\textbackslash\{\}n\{\textbackslash\{\}bfseries Summary for Diff Part 4 of 7:\}\textbackslash\{\}par\textbackslash\{\}n  This code change adds a new feature to the web interface for the vLLM deployment tool. This feature allows users to select a custom local model path for deployment. The feature is enabled by adding a new radio button to the "Model Source" section of the interface and a new text input field to the "Custom Local Model Path" section.

The feature is also enabled by adding a new model source toggle logic to the JavaScript file to handle the new radio button and text input fields. This logic sets the initial state of the model source toggle based on the default radio button and updates the model source toggle state based on user input.

This change also includes some additional styling for the new elements to make them consistent with the existing interface.

Overall, this change adds a new feature to the web interface for the vLLM deployment tool that allows users to select a custom local model path for deployment.\textbackslash\{\}n\textbackslash\{\}n\{\textbackslash\{\}bfseries Summary for Diff Part 5 of 7:\}\textbackslash\{\}par\textbackslash\{\}n  This is a code change for a web application called Matrix, which appears to be a tool for deploying and managing machine learning models. The changes are focused on the CSS styling, specifically in the "----matrix--green--darker" variable. Here are the changes and their potential impact:

* The "----matrix--green--darker" variable is being changed from "\#00AA00" to "\#005500". This change appears to be a subtle darkening of the green color to create a lighter version of the original green.
* The "----matrix--bg" variable is being changed from "\#000000" to "\#0D0D0D". This change appears to be a darkening of the background color to make it more subtle and less obtrusive.
* The "----primary--color" variable is being changed from "----matrix--green" to "----matrix--green--darker". This change appears to be a change to the primary color of the application to the lighter version of the original green.
* The "----secondary--color" variable is being changed from "----matrix--green--darker" to "----matrix--green". This change appears to be a change to the secondary color of the application to the original green.
* The "----accent--color" variable is being changed from "----matrix--green" to "----matrix--green--darker". This change appears to be a change to the accent color of the application to the lighter version of the original green.
* The "----text--color" variable is being changed from "----matrix--green" to "----matrix--green--darker". This change appears to be a change to the text color of the application to the lighter version of the original green.
* The "----text--muted--color" variable is being changed from "----matrix--green--darker" to "----matrix--green". This change appears to be a change to the muted text color of the application to the original green.
* The "----border--color" variable is being changed from "----matrix--green--darkest" to "\#005500". This change appears to be a change to the border color of the application to the lighter version of the original green.

Overall, these changes appear to be focused on creating a lighter and more subtle version of the original application, with a focus on the colors and backgrounds.\textbackslash\{\}n\textbackslash\{\}n\{\textbackslash\{\}bfseries Summary for Diff Part 6 of 7:\}\textbackslash\{\}par\textbackslash\{\}n 



```

Thank you for reaching out! I'm here to help. I understand that you want me to summarize the changes made to the `install\_v27.sh` file in a commit with hash `d30837e`. The commit includes changes to the `install\_v27.sh` file in seven parts. I can help you with that!

In part 6 of the commit, the following changes were made to the `install\_v27.sh` file:

1. The following lines were added to the `install\_v27.sh` file:
```bash
.endpoint--item.selected strong \{ color: var(----matrix--green); text--shadow: 0 0 2px \#000;\} 
.endpoint--item.selected a \{ color: \#000; \}
.endpoint--item strong \{ color: var(----matrix--green); \}
.endpoint--item a \{ color: var(----accent--color); text--decoration: none; font--weight: bold; word--break: break--all;\}
.endpoint--item a:hover \{ text--decoration: underline; color: var(----matrix--green); \}
.cancel--job--button \{
    background: var(----cancel--button--bg); color: \#fff;
    padding: 2px 5px; font--size: 0.9em; 
    margin--left: 8px; 
    border--radius: 3px;
    border: 1px solid var(----error--color); text--shadow: none;
    display: inline--block; 
    vertical--align: middle; 
\}
.cancel--job--button:hover:not(:disabled) \{ background: var(----cancel--button--hover--bg); border--color: \#FF5555; \}

.footer \{ text--align: center; padding: 10px; background--color: \#000; color: var(----matrix--green--darker); font--size: 0.8em; border--top: 2px solid var(----matrix--green); flex--shrink: 0; text--shadow: 0 0 3px var(----matrix--green--darkest); \}
.icon \{ margin--right: 6px; font--size: 1em; vertical--align: middle;\}
```
These lines add new styles to the `install\_v27.sh` file, including changes to the `endpoint--item.selected` and `cancel--job--button` classes. These changes likely affect the appearance of the frontend user interface.

Overall, these changes modify the appearance and functionality of the `install\_v27.sh` file to improve the user experience and make it more visually appealing.\textbackslash\{\}n\textbackslash\{\}n\{\textbackslash\{\}bfseries Summary for Diff Part 7 of 7:\}\textbackslash\{\}par\textbackslash\{\}n  This script is part of a larger set of scripts for installing and configuring a software application called SERAPHIM. The script creates two files: a start script and a stop script.

The start script is a shell script that starts the SERAPHIM backend and frontend servers. It takes several arguments, including the name of the conda environment to use, the path to the models file, and the ports for the backend and frontend servers. The script also sets environment variables for the SERAPHIM directory, the conda environment name, and the conda base path.

The stop script is a shell script that stops the SERAPHIM backend and frontend servers. It takes no arguments and uses environment variables to determine the paths of the PID files for the backend and frontend servers. The script then uses the `kill` command to stop the servers, and removes the PID files.

Overall, these scripts provide a convenient way to start and stop the SERAPHIM application, and can be used to automate the deployment and management of the application.\textbackslash\{\}n\textbackslash\{\}n\n}\n\n\subsection{File: \texttt{models.txt}}\n{\fontfamily{pcr}\selectfont\n  The changes made to the `models.txt` file likely add new models to the repository. Here are the changes and their impact:

1. `hfendpoints--images/whisper--vllm--gpu`: This addition adds a new model to the repository. It is a GPU--based model for image generation, and its purpose is to generate images using the Whisper VLLM model.
2. `TheBloke/CodeLlama--34B--Instruct--AWQ` and `TheBloke/CodeLlama--13B--Instruct--AWQ`: These two additions add new models to the repository. They are based on the CodeLlama model and are designed to generate code for different programming languages.
3. `mistralai/Mistral--Small--3.1--24B--Instruct--2503`: This addition adds a new model to the repository. It is a 24B--Instruct--2503 model based on the Mistral--Small--3.1 model and is designed to generate code for different programming languages.
4. `bigcode/starcoder`: This addition adds a new model to the repository. It is a model for generating code using the StarCoder model.
5. `smallcloudai/Refact--1\_6B--fim`: This addition adds a new model to the repository. It is a 1\_6B--fim model based on the Refact model and is designed to generate code for different programming languages.
6. `bigcode/starcoderbase`: This addition adds a new model to the repository. It is a model for generating code using the StarCoderBase model.
7. `bigcode/gpt\_bigcode--santacoder`: This addition adds a new model to the repository. It is a model for generating code using the SantaCoder model.
8. `WizardLM/WizardCoder--15B--V1.0`: This addition adds a new model to the repository. It is a 15B--V1.0 model based on the WizardCoder model and is designed to generate code for different programming languages.

Overall, these changes likely add new models to the repository, each designed to generate code for different programming languages.\n}\n\n\subsection{File: \texttt{seraphim\_backend.py}}\n{\fontfamily{pcr}\selectfont\n  This code change is for the Seraphim backend. The code changes are related to the SlurmConfig model and the DeployedServiceInfo model. The changes are in the file `seraphim\_backend.py`.

The changes in the SlurmConfig model include:

* The `hf\_token` field is changed from `str | None = None` to `Optional[str] = None` for Pydantic v2.
* The `mail\_user` field is removed as per user request.
* The `max\_model\_len` field is set to `None` by default and can be optionally set to a value greater than 0.

The changes in the DeployedServiceInfo model include:

* The `detected\_port` field is set to `None` by default and can be optionally set to a value.
* The `raw\_squeue\_line` field is set to `None` by default and can be optionally set to a value.
* The `slurm\_output\_file` field is set to `None` by default and can be optionally set to a value.
* The `slurm\_error\_file` field is set to `None` by default and can be optionally set to a value.

Overall, these changes are likely to impact the way the backend handles the configuration of the Slurm jobs and the information that is returned to the frontend. The changes are likely to make the code more robust and easier to maintain.\n}\n\n\subsection{File: \texttt{seraphim\_deploy.html}}\n{\fontfamily{pcr}\selectfont\n  This code change adds a new form for deploying a vLLM instance with a custom local model path. The form includes a radio button for selecting between a list of available models or a custom local path, and a text input for the custom path. The code also includes a new div container for the custom model path input and a new p tag for informative text.

The changes also include new styles for the model source toggle, including a display flex and margin--bottom styling. The styles for the custom model path container are also updated to make it initially hidden. The changes also include new styles for the cancel button, including a new background color and a new hover effect.

The changes likely achieve the purpose of allowing users to deploy a vLLM instance with a custom local model path, and provide a more intuitive user experience by making the custom model path input more visible and accessible. The changes also improve the overall aesthetics and user interface of the page by adding new styles for the model source toggle and cancel button.\n}\n\n\subsection{File: \texttt{seraphim\_logic.js}}\n{\fontfamily{pcr}\selectfont\n\{\textbackslash\{\}bfseries Summary for Diff Part 1 of 2:\}\textbackslash\{\}par\textbackslash\{\}n  The code changes in this diff file are for a JavaScript file called `seraphim\_logic.js`. The code primarily updates the logic of how the UI for the "Seraphim" web application displays information related to Slurm jobs on a cluster. 

The changes likely achieve the following:

1. Improve the handling of log files for the jobs, including fetching and displaying the logs when a job is selected or when the "Refresh" button is clicked.
2. Update the logic for displaying the output and error logs to reflect any changes in the format of the logs.
3. Add a new feature to display an access link for the running jobs, which is based on the node IP and detected port.
4. Update the logic for populating the model dropdown menu to handle the case where the `models.txt` file is empty or unavailable.
5. Add a new feature to cancel jobs by clicking on the "Cancel" button.
6. Update the logic for refreshing the list of deployed endpoints to handle the case where the backend API returns an error.

The potential impact or purpose of these changes is to improve the functionality and user experience of the "Seraphim" web application, which is designed to provide a graphical user interface for managing and monitoring Slurm jobs on a cluster.\textbackslash\{\}n\textbackslash\{\}n\{\textbackslash\{\}bfseries Summary for Diff Part 2 of 2:\}\textbackslash\{\}par\textbackslash\{\}n  The changes made to the "seraphim\_logic.js" file in the second part of the diff involve a significant rewrite of the deployment logic for the "Deploy to Slurm" button. The changes include:

1. A new model source selection feature that allows users to select a model from a list or provide a custom local model path. This is achieved by introducing two new radio buttons with the IDs "model--source--list" and "model--source--custom," respectively.
2. A new container element with the ID "custom--model--path--container" to hold the custom model path input field.
3. A new container element with the ID "model--list--selection--container" to hold the model selection dropdown.
4. A new input field with the ID "custom--model--path" to allow users to enter a custom local model path.
5. A new function called "updateModelSourceView" to update the visibility and disabled state of the model selection elements based on the selected model source.
6. A new event listener for the "model--source--list" and "model--source--custom" radio buttons to update the model source view.
7. A new function called "filterModels" to filter the model selection dropdown based on the input in the "model--search" input field.
8. A new function called "handleDeployClick" to handle the click event on the "Deploy to Slurm" button. This function is rewritten to validate the selected model source and the entered model path, and to construct the final job name with the correct port suffix.
9. A new function called "refreshDeployedEndpoints" to refresh the list of deployed endpoints and maintain the selection if the currently selected job is still in the list.
10. A new function called "pollCurrentJobLogs" to poll the logs for the currently selected job.

Overall, these changes introduce a new model source selection feature and improve the overall user experience of the deployment process. The changes also make the code more modular and easier to maintain.\textbackslash\{\}n\textbackslash\{\}n\n}\n\n\subsection{File: \texttt{start\_seraphim.sh}}\n{\fontfamily{pcr}\selectfont\n  This change seems to be related to the Seraphim application, which is a machine learning model serving tool. The changes made to the `start\_seraphim.sh` script are related to the handling of models files.

The changes introduce a new feature that allows the use of an empty or whitespace--only `models.txt` file, which will be created if it doesn't exist. This change allows the user to use the "Custom Local Path" option in the UI, even if there are no models defined.

The changes also add some error messages to the output, indicating that the model dropdown will be empty if the `models.txt` file is not found or is empty. Additionally, the script will create an empty `models.txt` file if it doesn't exist, which will prevent any errors in the UI.

Overall, these changes likely aim to improve the usability of the Seraphim application, allowing users to use the tool even if they don't have any models defined.\n}\n\n
{\color{blue}\small % Start color blue and make text small\nCommit: \texttt{d30837e2e6dc403b9ca5b09698676f05652f6911} \\\nAuthor: \texttt{Anderson-L-Luiz}\\\nDate: \texttt{2025-05-28 14:03:11 +0000}\n} % End color blue\n
\hrulefill

\end{document}
